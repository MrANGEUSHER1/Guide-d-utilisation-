\documentclass[12pt]{article}
\usepackage[utf8]{inputenc}
\usepackage[T1]{fontenc}
\usepackage[french]{babel}
\usepackage{geometry}
\usepackage{listings}
\usepackage{xcolor}
\usepackage{graphicx}
\usepackage{hyperref}
\usepackage{enumitem}
\usepackage{titlesec}

\geometry{a4paper, margin=2.5cm}

\lstset{
    basicstyle=\ttfamily,
    backgroundcolor=\color{gray!10},
    frame=single,
    breaklines=true
}

\title{Guide complet de la manipulation de notre application d’aide à l’archivage et à la consultation des projets des étudiants en fin de cycle.}
\author{Ange FOKAM}
\date{30/06/2025}

\begin{document}

\maketitle
\clearpage

\tableofcontents
\clearpage

% -------------------------------
\textbf{\textit{Pour les développeurs et les administrateurs système}}

\rule{\linewidth}{0.2pt}

\section*{À propos du guide :}
\addcontentsline{toc}{section}{À propos du guide}

\begin{itemize}[label=--]
    \item \textbf{Objectif du guide} : Explique comment déployer, manipuler et utiliser l'application.
    \item \textbf{Public cible} : Administrateurs système futurs et développeurs futurs.
    \item \textbf{Structure du document} : Deux sections, une pour le Matériel à disposer et l'autre pour le déploiement technique de l'application.
    \item \textbf{Prérequis généraux} : Accès réseau, identifiants, connaissances de base en informatique et en programmation.
\end{itemize}

\rule{\linewidth}{0.2pt}

\vfill
\rule{\linewidth}{0.2pt}
\textbf{\textit{Ce guide a pour objectif de rendre l’utilisation et la maintenance de l’application aussi claires et accessibles
que possible. En suivant les étapes décrites, les développeurs pourront exploiter pleinement le système tout en en assurant sa 
pérennité et son efficacité.}}

\newpage

% Introduction
{\fontsize{14}{16}\section*{Introduction}}
\addcontentsline{toc}{section}{Introduction}
Dans le monde du développement, les applications doivent être entretenues au quotidien.
À cet effet, elles doivent être conçues de manière à ce que les générations futures puissent
facilement les maintenir en activité. De plus, le développement des applications doit répondre
aux besoins de leurs utilisateurs.

De ce fait, nous avons mis en place une série d'étapes pour le lancement de notre application
à partir d’un poste personnel, afin de faciliter la compréhension et la maintenance par les
développeurs et administrateurs.

Bien que ce processus puisse sembler complexe, il vise à faciliter la prise en main et la
maintenance à long terme de l’application par les futurs intervenants, ainsi que la gestion
des projets de fin de cycle des étudiants.

Pour un déploiement complet de l'application, nous avons réparti la présentation en cinq
parties : la première présentera le matériel nécessaire, la deuxième expliquera l'installation
des environnements, la troisième montrera comment modifier les variables d'environnement, la
quatrième abordera le déploiement des serveurs, et la cinquième traitera de l'ouverture de l'application.

\vspace{0.5cm}
\rule{\linewidth}{0.2pt}

\newpage
% Guide de déploiement
{\fontsize{14}{16}\section*{Guide de déploiement de l’application}}
\addcontentsline{toc}{section}{Guide de déploiement de l’application}
\setcounter{subsection}{0} % On remet la numérotation des sections à zéro
\renewcommand\thesubsection{\arabic{subsection}}
\rule{\linewidth}{0.2pt}
\subsection{Matériel à disposer et prérequis}

Avant de commencer l'installation, assurez-vous d'avoir les éléments suivants :

\begin{itemize}[label=--]
    \item PHP (version 8.0 ou plus récente)
    \item Composer
    \item Angular
    \item Laravel 10+
    \item VS Code (de préférence comme éditeur de texte)
    \item MySQL (via XAMPP ou autre)
    \item Elasticsearch (version compatible)
    \item Git (avec un compte GitHub pour la collaboration)
\end{itemize}
\rule{\linewidth}{0.2pt}
\textbf{Configuration minimale recommandée :}

Pour ce qui est du matériel et les ressources nécessaires, vous devez procéder au déploiement.

\begin{itemize}[label=--]
    \item Processeur : 2 GHz double cœur ou plus
    \item RAM : 8 Go minimum
    \item Stockage : 30 Go d’espace libre
\end{itemize}
\rule{\linewidth}{0.2pt}
\textbf{Outils nécessaires :}

\begin{itemize}[label=--]
    \item Git
    \item Accès à Internet (FAI ou VPN)
    \item Navigateur web à jour (Chrome, Firefox, Edge, Opera, etc.)
\end{itemize}
\rule{\linewidth}{0.2pt}
\subsection{Étapes de déploiement}

Voici les étapes pour l’ouverture du projet (for windows and Linux):

\begin{enumerate}
    \item \textbf{Cloner le dépôt GitHub}

Lien du dépôt :
        \begin{lstlisting}
git clone https://github.com/Noobs440/UV_PROJET_AIGLE.git
        \end{lstlisting}
\rule{\linewidth}{0.2pt}
    \item \textbf{Configurer l’environnement}

Créer le fichier `.env` :
        \begin{lstlisting}
cp .env.example .env
        \end{lstlisting}
Ouvrez le fichier .env dans un éditeur de texte et configurez les variables d'environnement suivantes
en modifiant le fichier `.env`:
        \begin{lstlisting}
DB_CONNECTION=mysql
DB_HOST=127.0.0.1
DB_PORT=3306
DB_DATABASE=nom_de_votre_base
DB_USERNAME=nom_utilisateur
DB_PASSWORD=mot_de_passe
        \end{lstlisting}
\rule{\linewidth}{0.2pt}
    \item \textbf{Générer la clé d'application}

Générez une nouvelle clé d'application en exécutant la commande suivante :
        \begin{lstlisting}
php artisan key:generate
        \end{lstlisting}
\rule{\linewidth}{0.2pt}
    \item \textbf{Modification des variables d’environnement}
        \begin{figure}[h] 
            \centering 
            \includegraphics[width=0.58\textwidth]{./img/path.png} 
        \end{figure}

    \item \textbf{Configurer XAMPP et lancer Apache + MySQL}
        \begin{figure}[h!] 
            \centering 
            \includegraphics[width=0.7\textwidth]{./img/xampp.png} 
        \end{figure}
    \item \textbf{Créer la base de données sur MySQL}
    
\rule{\linewidth}{0.2pt}
    \item \textbf{Installer et configurer Elasticsearch} et remplacer le fichier `elasticsearch.yml` selon l'OS :

-Télécharger Elasticsearch depuis le site officiel.

-Extraire l'archive téléchargée dans un répertoire de votre choix.

-Copier le fichier elasticsearch.yml fourni dans le dépôt à l'emplacement de configuration d'Elasticsearch : 
        \begin{itemize}
            \item Windows :
            \begin{lstlisting}
copy config\elasticsearch\elasticsearch.yml C:\chemin\vers\elasticsearch\config\
            \end{lstlisting}
            \item Linux :
            \begin{lstlisting}
/bin/cp config/elasticsearch/elasticsearch.yml /chemin/vers/elasticsearch/config/
            \end{lstlisting}
        \end{itemize}
Configuration d'Elasticsearch :
    \begin{lstlisting}
SCOUT_DRIVER=Matchish\ScoutElasticSearch\Engines\ElasticSearchEngine
ELASTICSEARCH_HOST=http://localhost:9200
ELASTICSEARCH_USER=elastic
ELASTICSEARCH_PASSWORD=changeme
    \end{lstlisting}
\rule{\linewidth}{0.2pt}
    \item \textbf{Lancer Elasticsearch}

Accéder au répertoire d'installation d'Elasticsearch et exécuter :
        \begin{lstlisting}
cd chemin\vers\elasticsearch
# Windows
./bin/elasticsearch.bat
# Linux
./bin/elasticsearch
        \end{lstlisting}

Votre terminal affichera un résultat similaire à :
    \begin{figure}[h] 
        \centering 
        \includegraphics[width=0.8\textwidth]{./img/elastix.png} 
    \end{figure}

\rule{\linewidth}{0.2pt}
    \item \textbf{Lancer le backend Laravel}

Entrer le chemin d’accès au dossier backend dans votre terminal et exécuter :
        \begin{lstlisting}
cd chemin\vers\backend
php artisan migrate:refresh --seed
php artisan serve
        \end{lstlisting}
\rule{\linewidth}{0.2pt}
    \item \textbf{Lancer le frontend Angular}
       
Entrer le chemin d’accès au dossier frontend dans votre terminal et exécuter :
        \begin{lstlisting}
cd chemin\vers\frontend
ng serve
        \end{lstlisting}
L'applcication va commencer à builder et afficher à la fin de l'operation :
            \begin{figure}[h] 
                \centering 
                \includegraphics[width=0.8\textwidth]{./img/angular.png} 
            \end{figure}
Accéder à l'application sur :
    \begin{lstlisting}
http://localhost:4200/
    \end{lstlisting}
\end{enumerate}

\vspace{0.5cm}
\rule{\linewidth}{0.2pt}

{\fontsize{14}{16}\textbf{\textit{Problèmes courants :}}}

\begin{itemize}
    \item \textbf{Erreur de connexion à la base de données :} Vérifiez les identifiants dans `.env`.
    \item \textbf{Problème avec Elasticsearch :} Assurez-vous qu’il est bien démarré et que le port est correct.
\end{itemize}
\rule{\linewidth}{0.2pt}

\newpage
% Conclusion
{\fontsize{14}{16}\section*{Conclusion}}
\addcontentsline{toc}{section}{Conclusion}
En résumé, le processus de déploiement que nous avons détaillé vise à garantir une prise en main claire, structurée 
et pérenne de l'application. En suivant ces étapes, les développeurs et administrateurs actuels comme futurs pourront
assurer la continuité, la maintenance et l’évolution de l’application dans les meilleures conditions.
\end{document}