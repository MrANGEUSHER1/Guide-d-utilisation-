\documentclass[12pt]{article}
\usepackage[utf8]{inputenc}
\usepackage[T1]{fontenc}
\usepackage[french]{babel}
\usepackage{geometry}
\usepackage{listings}
\usepackage{xcolor}
\usepackage{graphicx}
\usepackage{hyperref}
\usepackage{enumitem}
\usepackage{titlesec}

\geometry{a4paper, margin=2.5cm}

\lstset{
    basicstyle=\ttfamily,
    backgroundcolor=\color{gray!10},
    frame=single,
    breaklines=true
}

\title{Guide complet de la manipulation de notre application}
\author{Ange Fokam}
\date{30/06/2025}

\begin{document}

\maketitle
\clearpage

\tableofcontents
\clearpage

% -------------------------------
%Pour les developpeur
\textbf{\textit{Pour les développeurs de administrateur système}}
\section*{Introduction générale}
\addcontentsline{toc}{section}{Introduction générale}

\begin{itemize}[label=--]
    \item \textbf{Objectif du guide} : Explique comment déployer, manipuler et utiliser l'application.
    \item \textbf{Public cible} : Administrateurs système futurs, développeurs futurs, utilisateurs finaux.
    \item \textbf{Structure du document} : Deux sections principales : déploiement technique, puis utilisation par les tiers.
    \item \textbf{Prérequis généraux} : Accès réseau, identifiants, connaissances de base en informatique et en programmation.
\end{itemize}

\section{Guide d’utilisation de l’application}

\subsection{Présentation de la page de connexion}
\begin{itemize}[label=--]
    \item Capture d’écran (facultatif)
    \item Description des champs requis : nom d’utilisateur, mot de passe, etc.
\end{itemize}

\subsection{Procédure de connexion}
\begin{itemize}[label=--]
    \item Étapes de connexion pour un utilisateur
    \item Gestion des erreurs de connexion
    \item Réinitialisation de mot de passe (si applicable)
\end{itemize}

\subsection{Soumission d’un projet}
\begin{itemize}[label=--]
    \item Accès à la fonctionnalité
    \item Champs à remplir
    \item Boutons/actions à déclencher
    \item Confirmation et suivi du projet
\end{itemize}

\subsection{Effectuer une requête}
\begin{itemize}[label=--]
    \item Accéder à la fonctionnalité
    \item Paramétrer la requête
    \item Valider et visualiser les résultats
    \item Sauvegarder ou exporter une requête (si disponible)
\end{itemize}

% -------------------------------
\section*{Annexes}
\addcontentsline{toc}{section}{Annexes}

\begin{itemize}[label=--]
    \item Glossaire des termes techniques
    \item Contacts techniques pour assistance
    \item Liens utiles (dépôts Git, documentation externe, etc.)
\end{itemize}

% -------------------------------
\section*{Conclusion}
\addcontentsline{toc}{section}{Conclusion}

\begin{itemize}[label=--]
    \item Résumé des étapes importantes
    \item Conseils pour une bonne utilisation
    \item Informations sur la mise à jour du guide
\end{itemize}

\vspace{0.5cm}

Ce guide a pour objectif de rendre l’utilisation et la maintenance de l’application aussi claire et fluide que possible. En respectant les étapes décrites, les utilisateurs comme les développeurs pourront tirer le meilleur parti du système tout en assurant sa pérennité.

\newpage

% Introduction
\section*{Introduction générale}
\addcontentsline{toc}{section}{Introduction générale}
Dans le monde du développement, les applications doivent sont de plus en utilisées au quotidien. 
À cet effet, elles doivent être conçues dans l'optique de répondre aux besoins de ses utilisateurs. 


De ce fait, nous avons jugée utile de developpeur un guide d'utilisation afin de rendre la prise en 
main de l'application fluide, par ses utilisateurs. Ainsi ils pourons mieux l'utiliser au quotidien,
sans sortir de son contexte, qui est de gérer l'archivage les projets de fin de cycle des étudiants. 

Pour une utilisation complète de l'application, nous avons présenté tout ce que les utilisateurs 
(étudiants, professeurs, visiteurs) doivent savoir et savoir faire sur l'application.

\vspace{0.5cm}
\rule{\linewidth}{0.2pt}

\rule{\linewidth}{0.2pt}
\end{document}